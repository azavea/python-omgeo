% Generated by Sphinx.
\def\sphinxdocclass{report}
\documentclass[letterpaper,10pt,english]{sphinxmanual}
\usepackage[utf8]{inputenc}
\DeclareUnicodeCharacter{00A0}{\nobreakspace}
\usepackage[T1]{fontenc}
\usepackage{babel}
\usepackage{times}
\usepackage[Bjarne]{fncychap}
\usepackage{longtable}
\usepackage{sphinx}
\usepackage{multirow}


\title{python-omgeo Documentation}
\date{September 17, 2012}
\release{1.5.0}
\author{Andrew Jennings, Joseph Tricarico, Justin Walgran}
\newcommand{\sphinxlogo}{}
\renewcommand{\releasename}{Release}
\makeindex

\makeatletter
\def\PYG@reset{\let\PYG@it=\relax \let\PYG@bf=\relax%
    \let\PYG@ul=\relax \let\PYG@tc=\relax%
    \let\PYG@bc=\relax \let\PYG@ff=\relax}
\def\PYG@tok#1{\csname PYG@tok@#1\endcsname}
\def\PYG@toks#1+{\ifx\relax#1\empty\else%
    \PYG@tok{#1}\expandafter\PYG@toks\fi}
\def\PYG@do#1{\PYG@bc{\PYG@tc{\PYG@ul{%
    \PYG@it{\PYG@bf{\PYG@ff{#1}}}}}}}
\def\PYG#1#2{\PYG@reset\PYG@toks#1+\relax+\PYG@do{#2}}

\def\PYG@tok@gd{\def\PYG@tc##1{\textcolor[rgb]{0.63,0.00,0.00}{##1}}}
\def\PYG@tok@gu{\let\PYG@bf=\textbf\def\PYG@tc##1{\textcolor[rgb]{0.50,0.00,0.50}{##1}}}
\def\PYG@tok@gt{\def\PYG@tc##1{\textcolor[rgb]{0.00,0.25,0.82}{##1}}}
\def\PYG@tok@gs{\let\PYG@bf=\textbf}
\def\PYG@tok@gr{\def\PYG@tc##1{\textcolor[rgb]{1.00,0.00,0.00}{##1}}}
\def\PYG@tok@cm{\let\PYG@it=\textit\def\PYG@tc##1{\textcolor[rgb]{0.25,0.50,0.56}{##1}}}
\def\PYG@tok@vg{\def\PYG@tc##1{\textcolor[rgb]{0.73,0.38,0.84}{##1}}}
\def\PYG@tok@m{\def\PYG@tc##1{\textcolor[rgb]{0.13,0.50,0.31}{##1}}}
\def\PYG@tok@mh{\def\PYG@tc##1{\textcolor[rgb]{0.13,0.50,0.31}{##1}}}
\def\PYG@tok@cs{\def\PYG@tc##1{\textcolor[rgb]{0.25,0.50,0.56}{##1}}\def\PYG@bc##1{\colorbox[rgb]{1.00,0.94,0.94}{##1}}}
\def\PYG@tok@ge{\let\PYG@it=\textit}
\def\PYG@tok@vc{\def\PYG@tc##1{\textcolor[rgb]{0.73,0.38,0.84}{##1}}}
\def\PYG@tok@il{\def\PYG@tc##1{\textcolor[rgb]{0.13,0.50,0.31}{##1}}}
\def\PYG@tok@go{\def\PYG@tc##1{\textcolor[rgb]{0.19,0.19,0.19}{##1}}}
\def\PYG@tok@cp{\def\PYG@tc##1{\textcolor[rgb]{0.00,0.44,0.13}{##1}}}
\def\PYG@tok@gi{\def\PYG@tc##1{\textcolor[rgb]{0.00,0.63,0.00}{##1}}}
\def\PYG@tok@gh{\let\PYG@bf=\textbf\def\PYG@tc##1{\textcolor[rgb]{0.00,0.00,0.50}{##1}}}
\def\PYG@tok@ni{\let\PYG@bf=\textbf\def\PYG@tc##1{\textcolor[rgb]{0.84,0.33,0.22}{##1}}}
\def\PYG@tok@nl{\let\PYG@bf=\textbf\def\PYG@tc##1{\textcolor[rgb]{0.00,0.13,0.44}{##1}}}
\def\PYG@tok@nn{\let\PYG@bf=\textbf\def\PYG@tc##1{\textcolor[rgb]{0.05,0.52,0.71}{##1}}}
\def\PYG@tok@no{\def\PYG@tc##1{\textcolor[rgb]{0.38,0.68,0.84}{##1}}}
\def\PYG@tok@na{\def\PYG@tc##1{\textcolor[rgb]{0.25,0.44,0.63}{##1}}}
\def\PYG@tok@nb{\def\PYG@tc##1{\textcolor[rgb]{0.00,0.44,0.13}{##1}}}
\def\PYG@tok@nc{\let\PYG@bf=\textbf\def\PYG@tc##1{\textcolor[rgb]{0.05,0.52,0.71}{##1}}}
\def\PYG@tok@nd{\let\PYG@bf=\textbf\def\PYG@tc##1{\textcolor[rgb]{0.33,0.33,0.33}{##1}}}
\def\PYG@tok@ne{\def\PYG@tc##1{\textcolor[rgb]{0.00,0.44,0.13}{##1}}}
\def\PYG@tok@nf{\def\PYG@tc##1{\textcolor[rgb]{0.02,0.16,0.49}{##1}}}
\def\PYG@tok@si{\let\PYG@it=\textit\def\PYG@tc##1{\textcolor[rgb]{0.44,0.63,0.82}{##1}}}
\def\PYG@tok@s2{\def\PYG@tc##1{\textcolor[rgb]{0.25,0.44,0.63}{##1}}}
\def\PYG@tok@vi{\def\PYG@tc##1{\textcolor[rgb]{0.73,0.38,0.84}{##1}}}
\def\PYG@tok@nt{\let\PYG@bf=\textbf\def\PYG@tc##1{\textcolor[rgb]{0.02,0.16,0.45}{##1}}}
\def\PYG@tok@nv{\def\PYG@tc##1{\textcolor[rgb]{0.73,0.38,0.84}{##1}}}
\def\PYG@tok@s1{\def\PYG@tc##1{\textcolor[rgb]{0.25,0.44,0.63}{##1}}}
\def\PYG@tok@gp{\let\PYG@bf=\textbf\def\PYG@tc##1{\textcolor[rgb]{0.78,0.36,0.04}{##1}}}
\def\PYG@tok@sh{\def\PYG@tc##1{\textcolor[rgb]{0.25,0.44,0.63}{##1}}}
\def\PYG@tok@ow{\let\PYG@bf=\textbf\def\PYG@tc##1{\textcolor[rgb]{0.00,0.44,0.13}{##1}}}
\def\PYG@tok@sx{\def\PYG@tc##1{\textcolor[rgb]{0.78,0.36,0.04}{##1}}}
\def\PYG@tok@bp{\def\PYG@tc##1{\textcolor[rgb]{0.00,0.44,0.13}{##1}}}
\def\PYG@tok@c1{\let\PYG@it=\textit\def\PYG@tc##1{\textcolor[rgb]{0.25,0.50,0.56}{##1}}}
\def\PYG@tok@kc{\let\PYG@bf=\textbf\def\PYG@tc##1{\textcolor[rgb]{0.00,0.44,0.13}{##1}}}
\def\PYG@tok@c{\let\PYG@it=\textit\def\PYG@tc##1{\textcolor[rgb]{0.25,0.50,0.56}{##1}}}
\def\PYG@tok@mf{\def\PYG@tc##1{\textcolor[rgb]{0.13,0.50,0.31}{##1}}}
\def\PYG@tok@err{\def\PYG@bc##1{\fcolorbox[rgb]{1.00,0.00,0.00}{1,1,1}{##1}}}
\def\PYG@tok@kd{\let\PYG@bf=\textbf\def\PYG@tc##1{\textcolor[rgb]{0.00,0.44,0.13}{##1}}}
\def\PYG@tok@ss{\def\PYG@tc##1{\textcolor[rgb]{0.32,0.47,0.09}{##1}}}
\def\PYG@tok@sr{\def\PYG@tc##1{\textcolor[rgb]{0.14,0.33,0.53}{##1}}}
\def\PYG@tok@mo{\def\PYG@tc##1{\textcolor[rgb]{0.13,0.50,0.31}{##1}}}
\def\PYG@tok@mi{\def\PYG@tc##1{\textcolor[rgb]{0.13,0.50,0.31}{##1}}}
\def\PYG@tok@kn{\let\PYG@bf=\textbf\def\PYG@tc##1{\textcolor[rgb]{0.00,0.44,0.13}{##1}}}
\def\PYG@tok@o{\def\PYG@tc##1{\textcolor[rgb]{0.40,0.40,0.40}{##1}}}
\def\PYG@tok@kr{\let\PYG@bf=\textbf\def\PYG@tc##1{\textcolor[rgb]{0.00,0.44,0.13}{##1}}}
\def\PYG@tok@s{\def\PYG@tc##1{\textcolor[rgb]{0.25,0.44,0.63}{##1}}}
\def\PYG@tok@kp{\def\PYG@tc##1{\textcolor[rgb]{0.00,0.44,0.13}{##1}}}
\def\PYG@tok@w{\def\PYG@tc##1{\textcolor[rgb]{0.73,0.73,0.73}{##1}}}
\def\PYG@tok@kt{\def\PYG@tc##1{\textcolor[rgb]{0.56,0.13,0.00}{##1}}}
\def\PYG@tok@sc{\def\PYG@tc##1{\textcolor[rgb]{0.25,0.44,0.63}{##1}}}
\def\PYG@tok@sb{\def\PYG@tc##1{\textcolor[rgb]{0.25,0.44,0.63}{##1}}}
\def\PYG@tok@k{\let\PYG@bf=\textbf\def\PYG@tc##1{\textcolor[rgb]{0.00,0.44,0.13}{##1}}}
\def\PYG@tok@se{\let\PYG@bf=\textbf\def\PYG@tc##1{\textcolor[rgb]{0.25,0.44,0.63}{##1}}}
\def\PYG@tok@sd{\let\PYG@it=\textit\def\PYG@tc##1{\textcolor[rgb]{0.25,0.44,0.63}{##1}}}

\def\PYGZbs{\char`\\}
\def\PYGZus{\char`\_}
\def\PYGZob{\char`\{}
\def\PYGZcb{\char`\}}
\def\PYGZca{\char`\^}
\def\PYGZsh{\char`\#}
\def\PYGZpc{\char`\%}
\def\PYGZdl{\char`\$}
\def\PYGZti{\char`\~}
% for compatibility with earlier versions
\def\PYGZat{@}
\def\PYGZlb{[}
\def\PYGZrb{]}
\makeatother

\begin{document}

\maketitle
\tableofcontents
\phantomsection\label{index::doc}



\chapter{omgeo}
\label{index:omgeo}\label{index:module-omgeo}\label{index:contents}\index{omgeo (module)}\index{Geocoder (class in omgeo)}

\begin{fulllineitems}
\phantomsection\label{index:omgeo.Geocoder}\pysiglinewithargsret{\strong{class }\code{omgeo.}\bfcode{Geocoder}}{\emph{sources=None}, \emph{preprocessors=None}, \emph{postprocessors=None}, \emph{waterfall=False}}{}
The base geocode class.  This class can be initialized with settings
for each geocoder and/or settings for the geocoder itself.
\begin{description}
\item[{Args:}] \leavevmode\begin{description}
\item[{sources (dict): a dictionary of GeocodeServiceConfig() parameters,}] \leavevmode
keyed by module name for the GeocodeService to use
ex: \{`esri\_wgs':\{\},
\begin{quote}
\begin{description}
\item[{`bing': \{`settings': \{\},}] \leavevmode
`preprocessors': {[}{]},
`postprocessors': {[}{]}\},

\end{description}

...\}
\end{quote}

\end{description}

preprocessors   -- list of universal preprocessors to use
postprocessors  -- list of universal postprocessors to use
waterfall       -- sets default for waterfall on geocode() method
\begin{quote}

(default False)
\end{quote}

\end{description}
\index{geocode() (omgeo.Geocoder method)}

\begin{fulllineitems}
\phantomsection\label{index:omgeo.Geocoder.geocode}\pysiglinewithargsret{\bfcode{geocode}}{\emph{pq}, \emph{waterfall=None}}{}~\begin{description}
\item[{Returns a dictionary including:}] \leavevmode\begin{itemize}
\item {} 
candidates - list of Candidate objects

\item {} 
upstream\_response\_info - list of UpstreamResponseInfo objects

\end{itemize}

\end{description}

pq          --  A PlaceQuery object (required).
waterfall   --  Boolean set to True if all geocoders listed should
\begin{quote}

be used to find results, instead of stopping after
the first geocoding service with valid candidates
(defaults to \textless{}Geocoder instance\textgreater{}.waterfall).
\end{quote}

\end{fulllineitems}

\index{set\_sources() (omgeo.Geocoder method)}

\begin{fulllineitems}
\phantomsection\label{index:omgeo.Geocoder.set_sources}\pysiglinewithargsret{\bfcode{set\_sources}}{\emph{sources}}{}
Creates GeocodeServiceConfigs from each str source
\begin{description}
\item[{sources --  list of source-settings pairs}] \leavevmode
ex. ``{[}{[}'EsriWGS', \{\}{]}, {[}'Nominatim', \{\}{]}{]}''

\end{description}

\end{fulllineitems}


\end{fulllineitems}



\chapter{places}
\label{index:module-omgeo.places}\label{index:places}\index{omgeo.places (module)}\index{Candidate (class in omgeo.places)}

\begin{fulllineitems}
\phantomsection\label{index:omgeo.places.Candidate}\pysiglinewithargsret{\strong{class }\code{omgeo.places.}\bfcode{Candidate}}{\emph{locator='`}, \emph{score=0}, \emph{match\_addr='`}, \emph{x=None}, \emph{y=None}, \emph{wkid=4326}, \emph{**kwargs}}{}
Class representing a candidate address returned from geocoders.
Accepts arguments defined below, plus informal keyword arguments.
\begin{description}
\item[{locator     -- Locator used for geocoding (default `')}] \leavevmode
We try to standardize this to `rooftop', `interpolated',
`postal\_specific', and `postal'.

\end{description}

score       -- Standardized score (default 0)
match\_addr  -- Address returned by geocoder (default `')
x           -- X-coordinate (longitude for lat-lon SRS) (default None)
y           -- Y-coordinate (latitude for lat-lon SRS) (default None)
wkid        -- Well-known ID for spatial reference system (default 4326)

Keyword arguments can be added in order to be able to use postprocessors
with API output fields are not well-fitting for one of the definitions
above
\begin{description}
\item[{c = Candidate(`US\_RoofTop', 91.5, `340 N 12th St, Philadelphia, PA, 19107',}] \leavevmode
`-75.16', `39.95', some\_key\_foo='bar')

\end{description}

\end{fulllineitems}

\index{PlaceQuery (class in omgeo.places)}

\begin{fulllineitems}
\phantomsection\label{index:omgeo.places.PlaceQuery}\pysiglinewithargsret{\strong{class }\code{omgeo.places.}\bfcode{PlaceQuery}}{\emph{query='`}, \emph{address='`}, \emph{city='`}, \emph{state='`}, \emph{postal='`}, \emph{country='`}, \emph{viewbox=None}, \emph{bounded=False}, \emph{**kwargs}}{}
Class representing an address or place passed to geocoders.
\begin{description}
\item[{query       --  A string containing the query to parse}] \leavevmode
and match to a coordinate on the map.
\emph{ex: ``340 N 12th St Philadelphia PA 19107''
or ``Wolf Building, Philadelphia''}

\item[{address     --  A string for the street line of an address.}] \leavevmode
\emph{ex: ``340 N 12th St''}

\item[{city        --  A string specifying the populated place for the address.}] \leavevmode
This commonly refers to a city, but may refer to a suburb
or neighborhood in certain countries.

\end{description}

state       --  A string for the state, province, territory, etc.
postal      --  A string for the postal / ZIP Code
country     --  A string for the country or region. Because the geocoder
\begin{quote}

uses the country to determine which geocoding service to use,
this is strongly recommended for efficency. ISO alpha-2 is
preferred, and is required by some geocoder services.
\end{quote}
\begin{description}
\item[{viewbox     --  A Viewbox object indicating the preferred area}] \leavevmode
to find search results (default None)

\item[{bounded     --  Boolean indicating whether or not to only}] \leavevmode
return candidates within the given Viewbox (default False)

\end{description}

user\_lat    --  A float representing the Latitude of the end-user.
user\_lon    --  A float representing the Longitude of the end-user.
user\_ip     --  A string representing the IP address of the end-user.
culture     --  Culture code to be used for the request (used by Bing).
\begin{quote}

For example, if set to `de', the country for a U.S. address
would be returned as ``Vereinigte Staaten Von Amerika''
instead of ``United States''.
\end{quote}

\end{fulllineitems}

\index{Viewbox (class in omgeo.places)}

\begin{fulllineitems}
\phantomsection\label{index:omgeo.places.Viewbox}\pysiglinewithargsret{\strong{class }\code{omgeo.places.}\bfcode{Viewbox}}{\emph{left=-180}, \emph{top=90}, \emph{right=180}, \emph{bottom=-90}, \emph{wkid=4326}}{}
Class representing a bounding box.
Defaults to maximum bounds for WKID 4326.

left    -- Minimum X value (default -180)
top     -- Maximum Y value (default 90)
right   -- Maximum X value (default 180)
bottom  -- Minimum Y value (default -90)
wkid    -- Well-known ID for spatial reference system (default 4326)
\index{convert\_srs() (omgeo.places.Viewbox method)}

\begin{fulllineitems}
\phantomsection\label{index:omgeo.places.Viewbox.convert_srs}\pysiglinewithargsret{\bfcode{convert\_srs}}{\emph{new\_wkid}}{}
Return a new Viewbox object with the specified SRS.

\end{fulllineitems}

\index{to\_bing\_str() (omgeo.places.Viewbox method)}

\begin{fulllineitems}
\phantomsection\label{index:omgeo.places.Viewbox.to_bing_str}\pysiglinewithargsret{\bfcode{to\_bing\_str}}{}{}
Convert Viewbox object to a string that can be used by Bing
as a query parameter.

\end{fulllineitems}

\index{to\_esri\_wgs\_json() (omgeo.places.Viewbox method)}

\begin{fulllineitems}
\phantomsection\label{index:omgeo.places.Viewbox.to_esri_wgs_json}\pysiglinewithargsret{\bfcode{to\_esri\_wgs\_json}}{}{}
Convert Viewbox object to a JSON string that can be used
by the ESRI World Geocoding Service as a parameter.

\end{fulllineitems}

\index{to\_mapquest\_str() (omgeo.places.Viewbox method)}

\begin{fulllineitems}
\phantomsection\label{index:omgeo.places.Viewbox.to_mapquest_str}\pysiglinewithargsret{\bfcode{to\_mapquest\_str}}{}{}
Convert Viewbox object to a string that can be used by
MapQuest as a query parameter.

\end{fulllineitems}


\end{fulllineitems}



\chapter{services}
\label{index:services}\label{index:module-omgeo.services.base}\index{omgeo.services.base (module)}\index{GeocodeService (class in omgeo.services.base)}

\begin{fulllineitems}
\phantomsection\label{index:omgeo.services.base.GeocodeService}\pysiglinewithargsret{\strong{class }\code{omgeo.services.base.}\bfcode{GeocodeService}}{\emph{preprocessors=None}, \emph{postprocessors=None}, \emph{settings=None}}{}
A tuple of classes representing the geocoders that will be used
to find addresses for the given locations
\index{geocode() (omgeo.services.base.GeocodeService method)}

\begin{fulllineitems}
\phantomsection\label{index:omgeo.services.base.GeocodeService.geocode}\pysiglinewithargsret{\bfcode{geocode}}{\emph{pq}}{}
Given an unprocessed PlaceQuery object, return a two-part tuple
including a post-processed list of Candidate objects 
and an UpstreamResponseInfo object if an API call was made.
\begin{description}
\item[{Preprocessor throws out request:}] \leavevmode
({[}{]}, None)

\item[{Postprocessor throws out some candidates:}] \leavevmode
({[}\textless{}Candidate obj\textgreater{}, \textless{}Candidate obj\textgreater{}{]}, \textless{}UpstreamResponseInfo obj\textgreater{})

\item[{Postprocessor throws out all candidates:}] \leavevmode
({[}{]}, \textless{}UpstreamResponseInfo obj\textgreater{})

\item[{An exception occurs while making the API call:}] \leavevmode
({[}{]}, \textless{}UpstreamResponseInfo obj\textgreater{})

\end{description}

\end{fulllineitems}


\end{fulllineitems}

\index{UpstreamResponseInfo (class in omgeo.services.base)}

\begin{fulllineitems}
\phantomsection\label{index:omgeo.services.base.UpstreamResponseInfo}\pysiglinewithargsret{\strong{class }\code{omgeo.services.base.}\bfcode{UpstreamResponseInfo}}{\emph{geoservice}, \emph{processed\_pq}, \emph{response\_code=None}, \emph{response\_time=None}, \emph{success=True}, \emph{errors=None}}{}
Class describing the API call result from an upstream provider.
For cleaning and consistency, set attributes using the given methods.

geoservice       -- name of the upstream provider used
processed\_pq     -- Processed PlaceQuery object

response\_code    -- HTTP response code (default None)
response\_time    -- time in milliseconds that it takes to get a
\begin{quote}

response (default None)
\end{quote}
\begin{description}
\item[{success          -- indicates if the API call was successful. A 200 response}] \leavevmode
with no candidates is still considered a success.
(default True)

\end{description}

errors           -- a list of human-readable error descriptions

\end{fulllineitems}

\phantomsection\label{index:module-omgeo.services}\index{omgeo.services (module)}\index{Bing (class in omgeo.services)}

\begin{fulllineitems}
\phantomsection\label{index:omgeo.services.Bing}\pysiglinewithargsret{\strong{class }\code{omgeo.services.}\bfcode{Bing}}{\emph{preprocessors=None}, \emph{postprocessors=None}, \emph{settings=None}}{}~\begin{description}
\item[{Class to geocode using Bing services:}] \leavevmode\begin{itemize}
\item {} 
\href{http://msdn.microsoft.com/en-us/library/ff701711.aspx}{Find a Location by Query}

\item {} 
\href{http://msdn.microsoft.com/en-us/library/ff701714.aspx}{Find a Location by Address}

\end{itemize}

\end{description}

api\_key --  The API key used to access Bing services.

\end{fulllineitems}

\index{CitizenAtlas (class in omgeo.services)}

\begin{fulllineitems}
\phantomsection\label{index:omgeo.services.CitizenAtlas}\pysiglinewithargsret{\strong{class }\code{omgeo.services.}\bfcode{CitizenAtlas}}{\emph{preprocessors=None}, \emph{postprocessors=None}, \emph{settings=None}}{}
Class to geocode using the Washington DC CitizenAtlas \textless{}\href{http://citizenatlas.dc.gov/newwebservices}{http://citizenatlas.dc.gov/newwebservices}\textgreater{}

\end{fulllineitems}

\index{EsriEUGeocodeService (class in omgeo.services)}

\begin{fulllineitems}
\phantomsection\label{index:omgeo.services.EsriEUGeocodeService}\pysigline{\strong{class }\code{omgeo.services.}\bfcode{EsriEUGeocodeService}}
Defaults for Esri EU Geocoders

As of 29 Dec 2011, the ESRI website claims to support Andorra, Austria, 
Belgium, Denmark, Finland, France, Germany, Gibraltar, Ireland, Italy,
Liechtenstein, Luxembourg, Monaco, The Netherlands, Norway, Portugal,
San Marino, Spain, Sweden, Switzerland, United Kingdom, and Vatican City.
\index{MAP\_FIPS\_TO\_ISO2 (omgeo.services.EsriEUGeocodeService attribute)}

\begin{fulllineitems}
\phantomsection\label{index:omgeo.services.EsriEUGeocodeService.MAP_FIPS_TO_ISO2}\pysigline{\bfcode{MAP\_FIPS\_TO\_ISO2}\strong{ = \{`MN': `MC', `EI': `IE', `DA': `DK', `AU': `AT', `GM': `DE', `AN': `AD', `SZ': `CH', `SP': `ES', `SW': `SE', `LS': `LI', `UK': `GB', `VT': `VC', `PO': `PT'\}}}
Map of FIPS to ISO-2 codes, if they are different.

\end{fulllineitems}

\index{SUPPORTED\_COUNTRIES\_FIPS (omgeo.services.EsriEUGeocodeService attribute)}

\begin{fulllineitems}
\phantomsection\label{index:omgeo.services.EsriEUGeocodeService.SUPPORTED_COUNTRIES_FIPS}\pysigline{\bfcode{SUPPORTED\_COUNTRIES\_FIPS}\strong{ = {[}'AN', `AU', `BE', `DA', `FI', `FR', `GM', `GI', `EI', `IT', `LS', `LU', `MN', `NL', `NO', `PO', `SM', `SP', `SW', `SZ', `UK', `VT'{]}}}
FIPS codes of supported countries

\end{fulllineitems}

\index{SUPPORTED\_COUNTRIES\_ISO2 (omgeo.services.EsriEUGeocodeService attribute)}

\begin{fulllineitems}
\phantomsection\label{index:omgeo.services.EsriEUGeocodeService.SUPPORTED_COUNTRIES_ISO2}\pysigline{\bfcode{SUPPORTED\_COUNTRIES\_ISO2}\strong{ = {[}'AD', `AT', `BE', `DK', `FI', `FR', `DE', `GI', `IE', `IT', `LI', `LU', `MC', `NL', `NO', `PT', `SM', `ES', `SE', `CH', `GB', `VC'{]}}}
ISO-2 codes of supported countries

\end{fulllineitems}


\end{fulllineitems}

\index{EsriGeocodeService (class in omgeo.services)}

\begin{fulllineitems}
\phantomsection\label{index:omgeo.services.EsriGeocodeService}\pysiglinewithargsret{\strong{class }\code{omgeo.services.}\bfcode{EsriGeocodeService}}{\emph{preprocessors=None}, \emph{postprocessors=None}, \emph{settings=None}}{}~\begin{description}
\item[{api\_key --  The API key used to access ESRI premium services.  If this}] \leavevmode
key is present, the object's endpoint will be set to use
premium tasks.

\end{description}

\end{fulllineitems}

\index{EsriNA (class in omgeo.services)}

\begin{fulllineitems}
\phantomsection\label{index:omgeo.services.EsriNA}\pysiglinewithargsret{\strong{class }\code{omgeo.services.}\bfcode{EsriNA}}{\emph{preprocessors=None}, \emph{postprocessors=None}, \emph{settings=None}}{}
Esri REST Geocoder for North America

\end{fulllineitems}

\index{EsriNAGeocodeService (class in omgeo.services)}

\begin{fulllineitems}
\phantomsection\label{index:omgeo.services.EsriNAGeocodeService}\pysigline{\strong{class }\code{omgeo.services.}\bfcode{EsriNAGeocodeService}}
Defaults for the EsriNAGeocodeService

\end{fulllineitems}

\index{EsriNASoap (class in omgeo.services)}

\begin{fulllineitems}
\phantomsection\label{index:omgeo.services.EsriNASoap}\pysiglinewithargsret{\strong{class }\code{omgeo.services.}\bfcode{EsriNASoap}}{\emph{preprocessors=None}, \emph{postprocessors=None}, \emph{settings=None}}{}
Use the SOAP version of the ArcGIS-10-style Geocoder for North America

\end{fulllineitems}

\index{EsriWGS (class in omgeo.services)}

\begin{fulllineitems}
\phantomsection\label{index:omgeo.services.EsriWGS}\pysiglinewithargsret{\strong{class }\code{omgeo.services.}\bfcode{EsriWGS}}{\emph{preprocessors=None}, \emph{postprocessors=None}, \emph{settings=None}}{}
Class to geocode using the ESRI World Geocoding service
\textless{}\href{http://geocode.arcgis.com/arcgis/geocoding.html}{http://geocode.arcgis.com/arcgis/geocoding.html}\textgreater{}.

This uses two endpoints -- one for single-line addresses,
and one for multi-part addresses.

\end{fulllineitems}

\index{EsriWGSSSL (class in omgeo.services)}

\begin{fulllineitems}
\phantomsection\label{index:omgeo.services.EsriWGSSSL}\pysiglinewithargsret{\strong{class }\code{omgeo.services.}\bfcode{EsriWGSSSL}}{\emph{preprocessors=None}, \emph{postprocessors=None}, \emph{settings=None}}{}
Class to geocode using the ESRI World Geocoding service over SSL
\textless{}\href{https://geocode.arcgis.com/arcgis/geocoding.html}{https://geocode.arcgis.com/arcgis/geocoding.html}\textgreater{}

\end{fulllineitems}

\index{MapQuest (class in omgeo.services)}

\begin{fulllineitems}
\phantomsection\label{index:omgeo.services.MapQuest}\pysiglinewithargsret{\strong{class }\code{omgeo.services.}\bfcode{MapQuest}}{\emph{preprocessors=None}, \emph{postprocessors=None}, \emph{settings=None}}{}
Class to geocode using MapQuest licensed services.

\end{fulllineitems}

\index{Nominatim (class in omgeo.services)}

\begin{fulllineitems}
\phantomsection\label{index:omgeo.services.Nominatim}\pysiglinewithargsret{\strong{class }\code{omgeo.services.}\bfcode{Nominatim}}{\emph{preprocessors=None}, \emph{postprocessors=None}, \emph{settings=None}}{}
Class to geocode using \href{http://open.mapquestapi.com/nominatim/}{Nominatim services hosted 
by MapQuest}.
\index{DEFAULT\_POSTPROCESSORS (omgeo.services.Nominatim attribute)}

\begin{fulllineitems}
\phantomsection\label{index:omgeo.services.Nominatim.DEFAULT_POSTPROCESSORS}\pysigline{\bfcode{DEFAULT\_POSTPROCESSORS}\strong{ = {[}\textless{}omgeo.processors.postprocessors.AttrFilter instance at 0x395cef0\textgreater{}, \textless{}omgeo.processors.postprocessors.AttrExclude instance at 0x395cf38\textgreater{}{]}}}
Postprocessors to use with this geocoder service, in order of desired execution.

\end{fulllineitems}

\index{DEFAULT\_PREPROCESSORS (omgeo.services.Nominatim attribute)}

\begin{fulllineitems}
\phantomsection\label{index:omgeo.services.Nominatim.DEFAULT_PREPROCESSORS}\pysigline{\bfcode{DEFAULT\_PREPROCESSORS}\strong{ = {[}\textless{}omgeo.processors.preprocessors.ReplaceRangeWithNumber instance at 0x395ce60\textgreater{}{]}}}
Preprocessors to use with this geocoder service, in order of desired execution.

\end{fulllineitems}


\end{fulllineitems}



\chapter{processors}
\label{index:module-omgeo.processors}\label{index:processors}\index{omgeo.processors (module)}\index{PostProcessor (class in omgeo.processors)}

\begin{fulllineitems}
\phantomsection\label{index:omgeo.processors.PostProcessor}\pysiglinewithargsret{\strong{class }\code{omgeo.processors.}\bfcode{PostProcessor}}{\emph{**kwargs}}{}
Takes, processes, and returns list of geocoding.places.Candidate objects.

\end{fulllineitems}

\index{PreProcessor (class in omgeo.processors)}

\begin{fulllineitems}
\phantomsection\label{index:omgeo.processors.PreProcessor}\pysiglinewithargsret{\strong{class }\code{omgeo.processors.}\bfcode{PreProcessor}}{\emph{**kwargs}}{}
Takes, processes, and returns a geocoding.places.PlaceQuery object.

\end{fulllineitems}

\phantomsection\label{index:module-omgeo.processors.preprocessors}\index{omgeo.processors.preprocessors (module)}\index{CancelIfPOBox (class in omgeo.processors.preprocessors)}

\begin{fulllineitems}
\phantomsection\label{index:omgeo.processors.preprocessors.CancelIfPOBox}\pysiglinewithargsret{\strong{class }\code{omgeo.processors.preprocessors.}\bfcode{CancelIfPOBox}}{\emph{**kwargs}}{}
Return False if the address is starts with any variation of ``PO Box''.
Otherwise, return original PlaceQuery.

\end{fulllineitems}

\index{CancelIfRegexInAttr (class in omgeo.processors.preprocessors)}

\begin{fulllineitems}
\phantomsection\label{index:omgeo.processors.preprocessors.CancelIfRegexInAttr}\pysiglinewithargsret{\strong{class }\code{omgeo.processors.preprocessors.}\bfcode{CancelIfRegexInAttr}}{\emph{regex}, \emph{attrs}, \emph{ignorecase=True}}{}
Return False if given regex is found in ANY of the given
PlaceQuery attributes, otherwise return original PlaceQuery instance.
In the event that a given attribute does not exist in the given
PlaceQuery, no exception will be raised.

regex         -- a regex string to match (represents what you do NOT want)
attrs         -- a list or tuple of strings of attribute names to look through
ignorecase    -- set to False for a case-sensitive match (default True)

\end{fulllineitems}

\index{CountryPreProcessor (class in omgeo.processors.preprocessors)}

\begin{fulllineitems}
\phantomsection\label{index:omgeo.processors.preprocessors.CountryPreProcessor}\pysiglinewithargsret{\strong{class }\code{omgeo.processors.preprocessors.}\bfcode{CountryPreProcessor}}{\emph{acceptable\_countries=}\optional{}, \emph{country\_map=\{\}}}{}~\begin{description}
\item[{acceptable\_countries -- A list of acceptable countries.}] \leavevmode
{[}{]} is used to indicate that all countries are acceptable.

An empty string is also an acceptable country. To require
a country, use the \emph{RequireCountry} preprocessor.

\item[{country\_map          -- A map of the input PlaceQuery.country property}] \leavevmode
to the country value accepted by the geocoding service.

Suppose that the geocoding service recognizes `GB', but not `UK',
and `US', but not `USA':
\begin{quote}

country\_map = \{`UK':'GB', `USA':'US'\}
\end{quote}

\end{description}

\end{fulllineitems}

\index{RequireCountry (class in omgeo.processors.preprocessors)}

\begin{fulllineitems}
\phantomsection\label{index:omgeo.processors.preprocessors.RequireCountry}\pysiglinewithargsret{\strong{class }\code{omgeo.processors.preprocessors.}\bfcode{RequireCountry}}{\emph{default\_country='`}}{}
Return False if no default country is set in first parameter.
Otherwise, return the default country if country is empty.

\end{fulllineitems}

\phantomsection\label{index:module-omgeo.processors.postprocessors}\index{omgeo.processors.postprocessors (module)}\index{AttrExclude (class in omgeo.processors.postprocessors)}

\begin{fulllineitems}
\phantomsection\label{index:omgeo.processors.postprocessors.AttrExclude}\pysiglinewithargsret{\strong{class }\code{omgeo.processors.postprocessors.}\bfcode{AttrExclude}}{\emph{bad\_values=}\optional{}, \emph{attr='locator'}, \emph{exact\_match=True}}{}
PostProcessor used to ditch results with unwanted attribute values.
\begin{description}
\item[{bad\_values   --  A list of values whose candidates we will}] \leavevmode
not accept results from (default {[}{]})

\end{description}

attr         --  The attribute type on which to filter
\begin{description}
\item[{exact\_match  --  True if attribute must match a bad value exactly.}] \leavevmode
False if the bad value can be a substring of the
attribute value. In other words, if our Candidate
attribute is `Postcode3' and one of the bad values
is `Postcode' because we want something more precise,
like `Address', we will not keep this candidate.

\end{description}

\end{fulllineitems}

\index{AttrFilter (class in omgeo.processors.postprocessors)}

\begin{fulllineitems}
\phantomsection\label{index:omgeo.processors.postprocessors.AttrFilter}\pysiglinewithargsret{\strong{class }\code{omgeo.processors.postprocessors.}\bfcode{AttrFilter}}{\emph{good\_values=}\optional{}, \emph{attr='locator'}, \emph{exact\_match=True}}{}
PostProcessor used to ditch results with unwanted attribute values.
\begin{description}
\item[{good\_values   --  A list of values whose candidates we will}] \leavevmode
accept results from (default {[}{]})

\end{description}

attr          --  The attribute type on which to filter
\begin{description}
\item[{exact\_match   --  True if attribute must match a good value exactly.}] \leavevmode
False if the attribute can be a substring in a
good value. In other words, if our Candidate
attribute is `US\_Rooftop' and one of the good\_values
is `Rooftop', we will keep this candidate.

\end{description}

\end{fulllineitems}

\index{AttrMigrator (class in omgeo.processors.postprocessors)}

\begin{fulllineitems}
\phantomsection\label{index:omgeo.processors.postprocessors.AttrMigrator}\pysiglinewithargsret{\strong{class }\code{omgeo.processors.postprocessors.}\bfcode{AttrMigrator}}{\emph{attr\_from}, \emph{attr\_to}, \emph{attr\_map=\{\}}, \emph{exact\_match=False}, \emph{case\_sensitive=False}}{}
PostProcessor used to migrate the given attribute
to another attribute.

attr\_from       -- Name of the input attribute
attr\_to         -- Name of the input attribute to overwrite
attr\_map        -- Dictionary of old names : new names.
exact\_match     -- Boolean
case\_sensitive  -- Boolean

\end{fulllineitems}

\index{AttrRename (class in omgeo.processors.postprocessors)}

\begin{fulllineitems}
\phantomsection\label{index:omgeo.processors.postprocessors.AttrRename}\pysiglinewithargsret{\strong{class }\code{omgeo.processors.postprocessors.}\bfcode{AttrRename}}{\emph{attr}, \emph{attr\_map=\{\}}, \emph{exact\_match=False}, \emph{case\_sensitive=False}}{}
PostProcessor used to rename the given attribute, with unspecified
attributes appearing at the end of the list.

attr            -- Name of the attribute
attr\_map        -- Dictionary of old names : new names.
exact\_match
case\_sensitive

\end{fulllineitems}

\index{AttrReverseSorter (class in omgeo.processors.postprocessors)}

\begin{fulllineitems}
\phantomsection\label{index:omgeo.processors.postprocessors.AttrReverseSorter}\pysiglinewithargsret{\strong{class }\code{omgeo.processors.postprocessors.}\bfcode{AttrReverseSorter}}{\emph{ordered\_values=}\optional{}, \emph{attr='locator'}}{}
PostProcessor used to sort by the given attributes in reverse order,
with unspecified attributes appearing at the end of the list.

This is good to use when a list has already been defined in a script
and you are too lazy to use the reverse() function, or don't want
to in order to maintain more readable code.
\begin{description}
\item[{ordered\_values   -- A list of values placed in the reverse }] \leavevmode
of the desired order.

\end{description}

\end{fulllineitems}

\index{AttrSorter (class in omgeo.processors.postprocessors)}

\begin{fulllineitems}
\phantomsection\label{index:omgeo.processors.postprocessors.AttrSorter}\pysiglinewithargsret{\strong{class }\code{omgeo.processors.postprocessors.}\bfcode{AttrSorter}}{\emph{ordered\_values=}\optional{}, \emph{attr='locator'}}{}
PostProcessor used to sort by a the given attribute, with unspecified
attributes appearing at the end of the list.

ordered\_values   --  A list of values placed in the desired order.
attr             --  The attribute on which to sort.

\end{fulllineitems}

\index{DupePicker (class in omgeo.processors.postprocessors)}

\begin{fulllineitems}
\phantomsection\label{index:omgeo.processors.postprocessors.DupePicker}\pysiglinewithargsret{\strong{class }\code{omgeo.processors.postprocessors.}\bfcode{DupePicker}}{\emph{attr\_dupes}, \emph{attr\_sort}, \emph{ordered\_list}, \emph{return\_clean=False}}{}
PostProcessor used to choose the best candidate(s)
where there are duplicates (or more than one result
that is very similar*) among high-scoring candidates,
such as an address.
\begin{itemize}
\item {} 
When comparing attribute values, case and commas do not count.

\end{itemize}

attr\_dupes      -- Property on which to look for duplicates.
attr\_sort       -- Property on which to sort using ordered\_list
ordered\_list    -- A list of property values, from most desirable
\begin{quote}

to least desirable.
\end{quote}
\begin{description}
\item[{return\_clean    -- Boolean indicating whether or not to}] \leavevmode
homogenize string values into uppercase
without commas.

\end{description}

\begin{tabulary}{\linewidth}{|L|L|L|}
\hline

match\_addr
 & 
score
 & 
locator
\\\hline

123 N Wood St
 & 
90
 & 
roof
\\\hline

123 S Wood St
 & 
90
 & 
address
\\\hline

123 N WOOD ST
 & 
77
 & 
address
\\\hline

123, S Wood ST
 & 
85
 & 
roof
\\\hline
\end{tabulary}


Above, the first two results have the highest scores. We could just
use those, because one of the two likely has the correct address.
However, the second result does not have the most precise location
for 123 S. Wood Street. While the fourth result does not score as
high as the first too, it's locator method is more desirable.
Since the addresses are the same, we can assume that the fourth result
will provide better data than the second.

We can get a narrowed list as described above by running the process()
method in the DupePicker() PostProcessor class as follows, assuming
that the ``candidates'' is our list of candidates:
\begin{quote}
\begin{description}
\item[{dp = DupePicker(}] \leavevmode
attr\_dupes='match\_addr',
attr\_sort='locator',
ordered\_list={[}'rooftop', `address\_point', `address\_range'{]})

\end{description}

return dp.process(candidates)
\end{quote}

Output:

\begin{tabulary}{\linewidth}{|L|L|L|}
\hline

match\_addr
 & 
score
 & 
locator
\\\hline

123 N Wood St
 & 
90
 & 
roof
\\\hline

123, S Wood ST
 & 
85
 & 
roof
\\\hline
\end{tabulary}


Output with return\_clean=True:

\begin{tabulary}{\linewidth}{|L|L|L|}
\hline

match\_addr
 & 
score
 & 
locator
\\\hline

123 N WOOD ST
 & 
90
 & 
roof
\\\hline

123 S WOOD ST
 & 
85
 & 
roof
\\\hline
\end{tabulary}


\end{fulllineitems}

\index{GroupBy (class in omgeo.processors.postprocessors)}

\begin{fulllineitems}
\phantomsection\label{index:omgeo.processors.postprocessors.GroupBy}\pysiglinewithargsret{\strong{class }\code{omgeo.processors.postprocessors.}\bfcode{GroupBy}}{\emph{attr='match\_addr'}}{}
Groups results by a certain attribute by choosing the
first occurrence in the list (this means you will want
to sort ahead of time).
\begin{description}
\item[{attr   --  The attribute on which to combine results}] \leavevmode
or a list or tuple of attributes where all
attributes must match between candidates.

\end{description}

\end{fulllineitems}

\index{GroupByMultiple (class in omgeo.processors.postprocessors)}

\begin{fulllineitems}
\phantomsection\label{index:omgeo.processors.postprocessors.GroupByMultiple}\pysiglinewithargsret{\strong{class }\code{omgeo.processors.postprocessors.}\bfcode{GroupByMultiple}}{\emph{attrs}}{}
Groups results by a certain attribute by choosing the
first occurrence in the list of candidates 
(this means you will want to sort ahead of time).

attrs   --  A list or tuple of attributes on which to combine results

\end{fulllineitems}

\index{LocatorFilter (class in omgeo.processors.postprocessors)}

\begin{fulllineitems}
\phantomsection\label{index:omgeo.processors.postprocessors.LocatorFilter}\pysiglinewithargsret{\strong{class }\code{omgeo.processors.postprocessors.}\bfcode{LocatorFilter}}{\emph{good\_locators=}\optional{}}{}
PostProcessor used to ditch results with lousy locators.
\begin{description}
\item[{good\_locators   --  A list of locators to}] \leavevmode
accept results from (default {[}{]})

\end{description}
\index{good\_locators (omgeo.processors.postprocessors.LocatorFilter attribute)}

\begin{fulllineitems}
\phantomsection\label{index:omgeo.processors.postprocessors.LocatorFilter.good_locators}\pysigline{\bfcode{good\_locators}\strong{ = {[}{]}}}
A list of Candidate.locator values that are good enough for what we need.

\end{fulllineitems}


\end{fulllineitems}

\index{LocatorSorter (class in omgeo.processors.postprocessors)}

\begin{fulllineitems}
\phantomsection\label{index:omgeo.processors.postprocessors.LocatorSorter}\pysiglinewithargsret{\strong{class }\code{omgeo.processors.postprocessors.}\bfcode{LocatorSorter}}{\emph{ordered\_locators=}\optional{}}{}
PostProcessor used to sort by locators
\index{ordered\_locators (omgeo.processors.postprocessors.LocatorSorter attribute)}

\begin{fulllineitems}
\phantomsection\label{index:omgeo.processors.postprocessors.LocatorSorter.ordered_locators}\pysigline{\bfcode{ordered\_locators}\strong{ = {[}{]}}}
A list of Candidate.locator values placed in the desired order.

\emph{Examples}:
\begin{quote}

{[}'rooftop', `address', `street', `city'{]}
\end{quote}

\end{fulllineitems}


\end{fulllineitems}

\index{ScoreSorter (class in omgeo.processors.postprocessors)}

\begin{fulllineitems}
\phantomsection\label{index:omgeo.processors.postprocessors.ScoreSorter}\pysiglinewithargsret{\strong{class }\code{omgeo.processors.postprocessors.}\bfcode{ScoreSorter}}{\emph{reverse=True}}{}
PostProcessor class to sort candidate scores.
\begin{description}
\item[{reverse  --  Boolean indicating if the scores should be sorted }] \leavevmode
descending (e.g. 100, 90, 80, ...) (default True)

\end{description}

\end{fulllineitems}

\index{SnapPoints (class in omgeo.processors.postprocessors)}

\begin{fulllineitems}
\phantomsection\label{index:omgeo.processors.postprocessors.SnapPoints}\pysiglinewithargsret{\strong{class }\code{omgeo.processors.postprocessors.}\bfcode{SnapPoints}}{\emph{distance=50}}{}
Chooses the first of two or more points where they are within the given
sphere-based great circle distance.

pnt 1       -- (x, y) tuple
pnt 2       -- (x, y) tuple    
distance    -- maximum distance (in metres) between two points in which
\begin{quote}

the the first will be kept and the second thrown out
(default 1).
\end{quote}

\end{fulllineitems}

\index{UseHighScoreIfAtLeast (class in omgeo.processors.postprocessors)}

\begin{fulllineitems}
\phantomsection\label{index:omgeo.processors.postprocessors.UseHighScoreIfAtLeast}\pysiglinewithargsret{\strong{class }\code{omgeo.processors.postprocessors.}\bfcode{UseHighScoreIfAtLeast}}{\emph{min\_score}}{}
Limit results to results with at least the given score,
if and only if one or more results has, at least, the
given score. If no results have at least this score,
all of the original results are returned intact.

\end{fulllineitems}



\chapter{Indices and tables}
\label{index:indices-and-tables}\begin{itemize}
\item {} 
\emph{genindex}

\item {} 
\emph{modindex}

\item {} 
\emph{search}

\end{itemize}


\renewcommand{\indexname}{Python Module Index}
\begin{theindex}
\def\bigletter#1{{\Large\sffamily#1}\nopagebreak\vspace{1mm}}
\bigletter{o}
\item {\texttt{omgeo}}, \pageref{index:module-omgeo}
\item {\texttt{omgeo.places}}, \pageref{index:module-omgeo.places}
\item {\texttt{omgeo.processors}}, \pageref{index:module-omgeo.processors}
\item {\texttt{omgeo.processors.postprocessors}}, \pageref{index:module-omgeo.processors.postprocessors}
\item {\texttt{omgeo.processors.preprocessors}}, \pageref{index:module-omgeo.processors.preprocessors}
\item {\texttt{omgeo.services}}, \pageref{index:module-omgeo.services}
\item {\texttt{omgeo.services.base}}, \pageref{index:module-omgeo.services.base}
\end{theindex}

\renewcommand{\indexname}{Index}
\printindex
\end{document}
